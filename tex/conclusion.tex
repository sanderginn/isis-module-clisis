\chapter{Conclusion}
\label{chapter:conclusion}
To conclude this research, we will assess the achieved results in relation to each research question. Research question~\ref{RQ3} can be answered first:

\begin{displayquote}
	\textit{Compared to the standard interface, how is performance affected when the user employs the new interface?}
\end{displayquote}

The empirical results as observed in section~\ref{subsection:timetrial_evaluation} show a significant increase in execution time of the experiment tasks when compared to the \acrshort{gui} execution time, but section~\ref{subsection:performancedifference} provides an analysis that is promising in terms of users becoming accustomed with the new user interface and therefore improving their execution time. Research has shown that \acrshortpl{gui} offer little to no performance improvement for expert users\cite{chen2007comparing}, which reaffirms the assumption that the target users will perform better as they grow acquainted with the new user interface. Due to the amount of speech that is unavoidable with a non-visual interface as proposed in this research, it is inevitable that an increase in execution time occurs. Moreover, improvements in execution time can still be achieved through several enhancements which will be layed out in section~\ref{section:futurework}. Considering the conditions and requirements which apply to a non-visual user interface, we therefore consider the decrease in performance to be within acceptable margins.\\
\newline
\noindent Next, research question~\ref{RQ2} can be addressed:

\begin{displayquote}
	\textit{When implementing a non-visual interface, can the integrity of the domain model be maintained while providing an effective and simple method of interaction?}
\end{displayquote}

Section~\ref{section:specificationrequirements} confirmed that all functionality derived from the detection stage as described in section~\ref{subsection:representation} has been implemented successfully. This verifies that the integrity of the domain model is intact, as all functionality is supported in the new user interface. The first part of research question~\ref{RQ2} can therefore be affirmed. If the new user interface provides an effective and simple method of interaction is debatable. Mainly, it is hard to determine what qualifies as 'effective' and 'simple', especially without feedback from users who are actually visually impaired, rather than simulating visual impairment. As the answer to research question~\ref{RQ3} states that the measured performance loss is expected and acceptable, we can conclude that the interaction can be considered effective. Whether it is simple is rather subjective. At the end of the time trial, the test subjects were all surprised with the improvements they made over the course of just a few tests, often citing the \textit{aha} moment they experienced during the tests. Therefore, we can conclude that it is possible to provide an effective and simple method of interaction while the integrity of the domain model is maintained.\\
\newline
\noindent 
Finally, research question~\ref{RQ1} can be answered:

\begin{displayquote}
	\textit{Can a content unaware graphical user interface be adapted to a non-visual interface so that visually impaired users are able to interact with the application?}
\end{displayquote}

The implementation as proposed in this research is abstracted from specific applications developed with Apache Isis and is therefore content unaware. The user interface can be deployed to any Apache Isis application and will work regardless of what functionality is offered by the application, barring any bespoke addons that are not part of the framework. Furthermore, the conclusions related to research question~\ref{RQ2} and~\ref{RQ3} confirm that it is possible to enable visually impaired users to interact with the application.\\
\newline
The findings reported in this paper affirm that it is possible to adapt a \acrshort{gui} to a non-visual user interface without compromising functionality. In terms of performance, some compromises have to be made; however, this is unavoidable when the user lacks the critical sense of eyesight.

\section{Future work}
\label{section:futurework}
While the new user interface as proposed in this research met the requirements to be qualified as successful, there is still a lot of improvement to be gained. The most critical issue that needs to be addressed is the way dropdown menus are represented in the new user interface, as described in section~\ref{subsection:performancedifference}. Future research will have to determine how the parameter choices should be displayed to improve usability. Additionally, tab completion of commands can be implemented to reduce the amount of typing required. Furthermore, aside from speech synthesis, the Web Speech API offers speech recognition as well. Implementing this in the new user interface would abolish the need for typing altogether and could, depending on the accuracy of the speech recognition, improve performance significantly when the user makes a lot of typing mistakes. Moreover, certain keywords can be replaced with auditory icons, e.g. crumpling paper on a delete action. This can reduce the amount of speech required for certain operations.

Other improvements do not improve user performance, but have a positive effect on the stability of the user interface. A major feature of AngularJS is its strong support for testing, which can enforce correct behaviour of the user interface. Due to the fact that we had no prior knowledge of AngularJS and the limited amount of time available in this research, our code is currently untested. For CLIsis to eventually become a mature user interface, it is imperative that a sufficient amount of test coverage is introduced.

In its current state, the user interface relies upon the \texttt{\$rootScope} to transfer data between application elements. This is not considered as desired behaviour, as there are more elegant solutions available such as transferring data through the use of events. An additional advantage of deprecating the use of \texttt{\$rootScope} is that it voids the need of the \texttt{rootScopeSanitiser} service. This undesired implementation can again be attributed to the lack of prior knowledge of AngularJS.

Finally, a great improvement would be the option to switch between CLIsis and the \acrshort{gui} while preserving context, as this would allow a visually impaired user to collaborate with or ask help from other users who do possess the ability to see. Currently, it will be a cumbersome task to share information with other users. If a switch between user interfaces can be made without losing active information, sharing and collaborating will become much more easy.

\section{Acknowledgements}
\label{section:acknowledgements}
I would like to than Dan for his continuous input throughout the months of this research, Maarten for the pleasant supervision of my project, Jeroen and Marc for enabling me to undertake this endeavour within Eurocommercial Properties and my girlfriend Janelle for pushing me to pursue my own project.